\documentclass[12pt]{article}
\usepackage{newtxtext}
\usepackage{graphicx} % Required for inserting images

\title{\textbf{NORMAS APA Y METODOLOGÍA DE LA INVESTIGACIÓN} \\ }
\author{
    Lourdes Adriana Pérez Barillas \\
    \textit{7690-19-5420 Universidad Mariano Gálvez} \\
    \texttt{Seminario de Tecnologías de información}
}
\date{27 de julio del 2024}
\begin{document}

\maketitle

\section{Resumen}
    \ Las Normas APA, establecidas por la American Psychological Association, fueron diseñadas para manejar la sobrecarga de información y asegurar una presentación uniforme de los trabajos académicos. Estas normas facilitan la comprensión y lectura de artículos científicos, lo que también contribuye a la estabilidad financiera de las revistas académicas. Además de las normas de presentación, el artículo aborda las metodologías de investigación, que son estructuras científicas para recopilar, analizar e interpretar datos. Existen varios tipos de metodologías: cualitativa, que explora fenómenos complejos mediante técnicas como entrevistas y observación participante; cuantitativa, que recolecta y analiza datos numéricos para identificar patrones y probar hipótesis; métodos mixtos, que combinan ambos enfoques para una comprensión más completa; investigación descriptiva, que describe características de una población sin manipular variables; investigación experimental, que manipula variables para observar efectos y establecer relaciones causales; e investigación correlacional, que identifica relaciones entre variables sin probar causalidad. Estas metodologías permiten abordar preguntas de investigación desde múltiples perspectivas, proporcionando una base sólida para la validación de hipótesis y la comprensión profunda de diversos fenómenos. \\



\section{
    Palabras Clave
}
\begin{itemize}
    \item \textbf{Comprensión}
    \item \textbf{Lectura eficiente}
    \item \textbf{Presentación uniforme}
    \item \textbf{Análisis}
    \item \textbf{Información científica}
\end{itemize}

\section{Desarrollo del tema}
    \textbf{NORMAS APA} \\ 
        American Psychological Association surgió (APA) surgió cuando un grupo de psicólogos, antropólogos y administradores empresariales se reunieron para establecer un conjunto de directrices simples que codifican los muchos componentes de la escritura científica. Su objetivo era aumentar la comprensión y facilitar la lectura de los trabajos académicos. El propósito de las normas APA era manejar la sobrecarga de información en el creciente número de artículos científicos, haciendo que los lectores tuvieran una experiencia más eficiente y fluida. Además las normas APA buscaban univormizar la presentación de los trabajos académicos para asegurar que fueran más fáciles de leer y evaluar, lo cual también contribuiría a la estabilidad financiera de las revistas académicas. (EFPSA, 2012).

        Las normas APA, diseñadas para manejar la sobrecarga de información y asegurar una presentación uniforme de los trabajos académicos, han tenido un impacto significativo al permitir una experiencia de lectura más eficiente y fluida, y al estabilizar financieramente las revistas académicas (EFPSA, 2012). \\ 
    \\ \textbf{METODOLOGÍAS DE INVESTIGACION} \\ 
        Es una estructura y acercamiento científico usado para recopilar, analizar e interpretar datos cuantitativos o cualitativos para responder preguntas de investigación. Históricamente, las ciencias naturales fueron pioneras en el desarrollo de métodos sistemáticos de investigación para validar hipótesis mediante experimentos controlados. Posteriormente, las ciencias sociales adoptaron y adaptaron estas metodologías para estudiar fenómenos humanos y sociales, desarrollando técnicas cualitativas como la etnografía y los estudios de caso, que permiten una comprensión más profunda de contextos y comportamientos complejos (Research Prospect, 2023).

        Los tipos de metodólogía de la investigación son:
        \begin{itemize}
        \item Investigación Cualitativa: Esta metodología se enfoca en comprender fenómenos complejos desde una perspectiva más profunda y subjetiva. Incluye técnicas como entrevistas, observación participante y análisis de contenido, y se utiliza para explorar experiencias, comportamientos y significados sociales.

        \item Investigación Cuantitativa: Se centra en la recolección y análisis de datos numéricos para identificar patrones y probar hipótesis. Métodos comunes incluyen encuestas, experimentos y análisis estadísticos.
        
        \item Métodos Mixtos: Combinan elementos de la investigación cualitativa y cuantitativa para aprovechar las fortalezas de ambos enfoques. Esta metodología permite una comprensión más completa y robusta de los problemas de investigación.
        
        \item Investigación Descriptiva: Se utiliza para describir características de una población o fenómeno sin manipular variables. Se basa en observaciones detalladas y puede incluir encuestas y estudios de caso.
        
        \item Investigación Experimental: Implica la manipulación de una o más variables independientes para observar su efecto en una variable dependiente. Es útil para establecer relaciones de causa y efecto.
        
        \item Investigación Correlacional: Busca identificar relaciones entre dos o más variables sin manipularlas. Aunque puede mostrar asociaciones, no puede probar causalidad.
    \end{enumerate}    
\section{
    Observaciones y comentarios
}
    El artículo tiene como objetivo dar a entender la necesidad de estructurar y estandarizar las investigaciones científicas para una mayor comprensión de parte del lector.
\section{
    Conclusiones
}
    La necesidad de enfocar en una metodología de investigación o norma APA es impactar en el entendimiento de usuarios lectores y escritores. Para las metodologías de investigación es importante se pueda establecer cómo se realizará el estudio y qué se pretende obtener.
\section{
    Bibliografía
}
    \ EFPSA (2012, 10 de julio). The origins of APA style (and why there are so many rules). Recuperado de https://blog.efpsa.org/2012/07/10/the-origins-of-apa-style-and-why-there-are-so-many-rules \\ \\
    \  Research Prospect. (2023, 10 de julio). The Ultimate Guide To Research Methodology. Recuperado de https://researchprospect.ca/the-ultimate-guide-to-research-methodology \\ \\
    \ ResearchMethodology.org. (2024). Types of Research Methods. Recuperado de researchmethodology.org. \\

\end{document}
